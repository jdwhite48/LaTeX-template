%%%%%%%%%%%%%%%%%%%%%%%%%%%%%%%%%%%%%%%%%%%%%%%%%%%%%%%%%%%%%%%%%%%%
%   Created By:     Jacob White
%   Email:          white570@purdue.edu
%
%%%%%%%%%%%%%%%%%%%%%%%%%%%%%%%%%%%%%%%%%%%%%%%%%%%%%%%%%%%%%%%%%%%%

\documentclass{article}
\usepackage[fontsize=12pt]{scrextend}   % specify 12pt font
\usepackage[margin=1in]{geometry}     % specify 1in margins

%%% Common template packages (must come before macros.tex)
%%% Formatting packages
\usepackage[utf8]{inputenc}
\usepackage{footnote}
\usepackage{xcolor}     % color names
\usepackage{graphicx}
\usepackage{url}        % URLs
\usepackage{hyperref}   % named URLs 
\usepackage{enumitem}   % better enumerations
\usepackage{multirow}   % multi-row math tables
\usepackage{multicol}   % multi-column pages and math tables
\usepackage{soul}       % strikethrough
\usepackage{parskip}    % Use decent \parskip. Also disables \parindent as default (must manually specify)

%%% Math packages
\usepackage{amsmath}
\usepackage{amssymb}    % \mathcal, \mathbb, etc.
\usepackage{amsthm}     % math theorem enumerations
\usepackage{amsfonts}
\usepackage{mathtools}  % \coloneqq, ...
\usepackage{bbm}
\usepackage{mathpartir}

%%% CS packages
\usepackage{ifthen}
\usepackage{xifthen}
\usepackage{listings}   % code

%%% Macros, environments, and other constructions
%%%%%%%%%%%%%%%%%%%%%%%%%%%%%%%%%%%%%%%%%%%%%%%%%%%%%%%%%%%%%%%%%%%%
%   Created By:     Jacob White
%   Email:          white570@purdue.edu
%   Last Updated:   27 August 2021
%
%   Sources of inspiration:
%       - Christina Garman's Practical & Applied Cryptography midterm template
%       - Reed Oei's math homework template (https://github.com/ReedOei/LaTeX)
%       - K A Gillow's University of Oxford, Mathematical Institute LaTeX Problem Sheet (https://www.maths.ox.ac.uk/members/it/faqs/latex/problem-sheets)
%       - Sources for specific commands may be referenced inline.
%
%%%%%%%%%%%%%%%%%%%%%%%%%%%%%%%%%%%%%%%%%%%%%%%%%%%%%%%%%%%%%%%%%%%%

\usepackage{amsmath}
\usepackage{amssymb}
\usepackage{amsthm}
\usepackage{footnote}
\usepackage{enumitem}

\usepackage{ifthen}
\usepackage{xifthen}
\usepackage{calc}

\usepackage{xspace}

%%% Change hyperref's colors
\hypersetup{            
  colorlinks=true,
  linkcolor=black,   % internal links (e.g. equation and theorem labels)
  urlcolor=cyan,     % \hyperref and \url
  filecolor=red,
  citecolor=black,
}

%%% Author notes
%\newcommand{\jw}[1]{\dtcolornote[JW]{cyan}{#1}}

%%% Change color of footnote links to black
\renewcommand{\thefootnote}{\textcolor{black}{\arabic{footnote}}}

%%% From: https://en.wikibooks.org/wiki/LaTeX/Source_Code_Listings#Automating_file_inclusion
\newcommand{\includecode}[2][c]{\lstinputlisting[language=#1, caption=#2, basicstyle=\normalsize\ttfamily]{#2}}

%%% Better paragraph indents
%\setlength{\parskip}{0em}       % parskip import does this already -- adds spacing after paragraph "by about the length of a lowercase 'x'"
\AtBeginDocument{\setlength{\parindent}{15pt}}     % Re-enable \parindent as default (equiv. spacing for 11pt) even if parskip imported

%%%%%%%%%%%%%%%%%%%
% Math macros     %
%%%%%%%%%%%%%%%%%%%

%%% General Arithmetic
\newcommand     {\nthroot}[2]{\ensuremath{\sqrt[\leftroot{-1}\uproot{1}\scriptstyle #1]{#2}}}
\newcommand     {\abs}[1]{\ensuremath{\left| #1 \right|}}
\newcommand     {\inverse}[1]{\ensuremath{\frac{1}{#1}}\xspace}
\newcommand     {\half}{\inverse{2}}
\newcommand     {\third}{\inverse{3}}
\newcommand     {\fourth}{\inverse{4}}
%%% Groups, Fields, and Rings
\newcommand     {\group}{\ensuremath{\mathbb{G}}\xspace}
\newcommand     {\G}{\group}
\newcommand     {\nat}{\ensuremath{\mathbb{N}}\xspace}
\newcommand     {\naturals}{\nat}
\newcommand     {\N}{\nat}
\newcommand     {\ints}{\ensuremath{\mathbb{Z}}\xspace}
\newcommand     {\integers}{\int}
\newcommand     {\Z}{\int}
\newcommand     {\Zmod}[1]{\ensuremath{\Z_{#1}}\xspace}
\newcommand     {\rat}{\ensuremath{\mathbb{Q}}\xspace}
\newcommand     {\rationals}{\rat}
\newcommand     {\Q}{\rat}
\newcommand     {\real}{\ensuremath{\mathbb{R}}\xspace}
\newcommand     {\reals}{\real}
\newcommand     {\R}{\real}
\newcommand     {\complex}{\ensuremath{\mathbb{C}}\xspace}
\newcommand     {\C}{\complex}
\newcommand     {\field}{\ensuremath{\mathbb{F}}\xspace}
\newcommand     {\F}{\field}
\newcommand     {\ring}{\ensuremath{\mathcal{R}}\xspace}
\newcommand     {\rng}{\ring}
\newcommand     {\inftypt}{\ensuremath{\mathcal{O}}\xspace}
%%% Set notation
\newcommand     {\powerset}[1]{\ensuremath{\mathcal{P}(#1)}\xspace}
\newcommand     {\set}[1]{\ensuremath{\left\{ #1 \right\}}\xspace}
%%% Change to colon if you prefer that conditional notation over a vertical bar (\mid)
%\newcommand {\suchthat}{\xspace:\xspace}
\newcommand     {\suchthat}{\xspace\mid\xspace}
\newcommand     {\given}{\suchthat}
%%% Prevents overloading strikethrough macro if the document imports soul package
\ifdefined\st\else\newcommand{\st}{\suchthat}\fi
\newcommand     {\cset}[2]{\set{#1 \suchthat #2}}  % a conditional notation to define sets
\newcommand     {\lset}[2]{\set{#1,\ldots,#2}} % set {from,...,to}
\renewcommand   {\complement}[1]{\ensuremath{\overline{#1}}}
\newcommand     {\setcomplement}[2]{\ensuremath{#1\setminus{#2}}}
\newcommand     {\minuscomplement}[2]{\ensuremath{#1-#2}}
%%% Logic
%\renewcommand   {\implies}{ \longrightarrow }
%%% Custom functions
\renewcommand   {\max}[1]{\ensuremath{\text{{\upshape{max}}}\{#1\}}\xspace}
\renewcommand   {\min}[1]{\ensuremath{\text{{\upshape{min}}}\{#1\}}\xspace}
%%% Custom operators
\newcommand     {\bigcdot}{\ensuremath{\bullet}\xspace}
%%% Custom equivalence Relations
\renewcommand   {\approx}{\raise.17ex\hbox{\ensuremath{\scriptstyle\sim}}}
\renewcommand   {\gets}{\coloneqq}
%%% Relations-checking operators
\newcommand     {\iseq}{\overset{?}{=}}
\newcommand     {\isequiv}{\overset{?}{\equiv}}
\newcommand     {\iscong}{\overset{?}{\cong}}
\newcommand     {\isapprox}{\overset{?}{\approx}}
\newcommand     {\issim}{\overset{?}{\isapprox}}
\newcommand     {\issimeq}{\overset{?}{\simeq}}
\newcommand     {\isdoteq}{\overset{?}{\doteq}}
\newcommand     {\islt}{\overset{?}{<}}
\newcommand     {\isgt}{\overset{?}{>}}
\newcommand     {\isleq}{\overset{?}{\leq}}
\newcommand     {\isgeq}{\overset{?}{\geq}}
\newcommand     {\issubset}{\overset{?}{\subset}}
\newcommand     {\issupset}{\overset{?}{\supset}}
\newcommand     {\issubsetq}{\overset{?}{\subseteq}}
\newcommand     {\issupsetq}{\overset{?}{\supseteq}}
\newcommand     {\isprec}{\overset{?}{\prec}}
\newcommand     {\ispreceq}{\overset{?}{\preceq}}
\newcommand     {\issucc}{\overset{?}{\succ}}
\newcommand     {\issucceq}{\overset{?}{\succeq}}
\newcommand     {\ispropto}{\overset{?}{\propto}}
%%% Statistics
\renewcommand   {\Pr}[1]{\ensuremath{\mbox{Pr}\left[#1\right]}}
%%% Calculus
\newcommand     {\approaches}{\rightarrow}

%%%%%%%%%%%%%%%%%%%
% CS Macros       %
%%%%%%%%%%%%%%%%%%%
%%% General
\newcommand     {\alg}[1]{\ensuremath{\mathsf{#1}}\xspace}
\newcommand     {\var}[1]{\ensuremath{\mathsf{#1}}\xspace}
\newcommand     {\true}{\var{T}}
\newcommand     {\false}{\var{F}}
\newcommand     {\boolset}{\set{\true,\false}}
\newcommand     {\bitset}{\ensuremath{\set{0,1}*}\xspace}
%%% Reducibility
\newcommand     {\mapreducesto}{\leq_m}
\newcommand     {\mapreducesfrom}{\geq_m}
\newcommand     {\polyreducesto}{\leq_p}
\newcommand     {\polyreducesfrom}{\geq_p}
%%% Turing Machines
\newcommand     {\yields}{\rightarrow_M}
\newcommand     {\yieldsto}[1][*]{\overset{#1}{\rightarrow}_M}
\newcommand     {\blanksymbol}{%
  \ooalign{%
    \relax\cr%
    \noalign{\vskip.2ex}%
    \textvisiblespace\cr%
    \noalign{\vskip-.2ex}%
    \hphantom{~}\cr%   Width of a space character
    \vphantom{Xg}\cr%  Height of X, depth of g
  }\xspace%
}
%%% Grammars + Languages + Strings
\newcommand   {\derives}{\rightarrow}
\newcommand   {\derivesto}[1][*]{\overset{#1}{\rightarrow}}
%%% From: https://tex.stackexchange.com/questions/41357/replace-cmtt10-char32-visible-cup-space-with-something-more-gentle/41445#41445
\newcommand   {\emptystring}{\ensuremath{\epsilon}\xspace}
\newcommand   {\emptysymbol}{\emptystring}
\newcommand   {\langconcat}[2]{#1 \circ #2}
\newcommand   {\kleenestar}[1]{{#1}^*}
%%% Change to comma if you prefer that string concatenation over 2 vertical bars
%\newcommand  {\strconcat}{\xspace\ensuremath{,}\xspace}
\newcommand   {\strconcat}{\xspace\ensuremath{\Vert}\xspace}
\newcommand   {\concat}{\strconcat}
%%% Graphs
\newcommand   {\pathto}{\rightsquigarrow}
\newcommand   {\dirpath}{\pathto}
\newcommand   {\dirpathto}{\pathto}
\newcommand   {\undirpath}{\leftrightsquigarrow}
\newcommand   {\undirpathto}{\undirpath}
\newcommand   {\undirpathfrom}{\undirpath}
%%% Cryptography
\newcommand   {\secparam}{\ensuremath{\lambda}\xspace}
\newcommand   {\crs}{\var{crs}}
\newcommand   {\rel}[1]{\ensuremath{R_{\mathsf{#1}}}\xspace}
\newcommand   {\com}{\var{com}}


%%%%%%%%%%%%%%%%%%%
% Theorems etc.   %
%%%%%%%%%%%%%%%%%%%
\newcounter{problem}
\renewcommand{\theproblem}{\arabic{problem}}    
\newenvironment{problem}[1][] { % Optionally include descriptor and make following text italics by default
    \stepcounter{problem} \noindent \underline{\textbf{Problem \theproblem}}:
    \ifx\newenvironment#1\newenvironment\else\space(#1)\fi \space\itshape 
} {\vspace{1em}}

\newenvironment{solution} {\par} {\newpage}
%\renewenvironment{proof} {\vspace{0.5em} \noindent \underline{\textbf{Pf}}:} {\hfill \qedsymbol \vspace{0.5em}}      % Default proof, but with _*Pf:*_ instead
\newenvironment{theoremproof} {\vspace{0.5em} \noindent \textit{Proof \thetheorem.}} {\hfill \qedsymbol \vspace{0.5em}}      % Default proof, numbered by theorem
\newenvironment{corollaryproof} {\vspace{0.5em} \noindent \textit{Proof \thecorollary.}} {\hfill \qedsymbol \vspace{0.5em}}      % Default proof, numbered by corollary
\newenvironment{lemmaproof} {\vspace{0.5em} \noindent \textit{Proof \thelemma.}} {\hfill \qedsymbol \vspace{0.5em}}      % Default proof, numbered by lemma
\newenvironment{proofsketch} {\vspace{0.5em} \noindent \textit{Proof (Sketch).}} {\vspace{0.5em}}      % Default proof, but as an informal description

\ifdefined\definition\else\newtheorem{definition}{Definition}\fi
\newtheorem*{definition*}{Definition}
\ifdefined\theorem\else\newtheorem{theorem}{Theorem}[problem]\fi
\newtheorem*{theorem*}{Theorem}
%%% TODO: How do I want to number assumptions? Is a separate counter sufficient?
\ifdefined\assumption\else\newtheorem{assumption}{Assumption}[problem]\fi
\newtheorem*{assumption*}{Assumption}
\ifdefined\corollary\else\newtheorem{corollary}{Corollary}[theorem]\fi
\newtheorem*{corollary*}{Corollary}
\ifdefined\lemma\else\newtheorem{lemma}{Lemma}[theorem]\fi
\newtheorem*{lemma*}{Lemma}
\ifdefined\claim\else\newtheorem{claim}{Claim}[problem]\fi
\newtheorem*{claim*}{Claim}
\ifdefined\observation\else\newtheorem*{observation}{Observation}\fi
\ifdefined\proposition\else\newtheorem*{proposition}{Proposition}\fi
\ifdefined\fact\else\newtheorem*{fact}{Fact}\fi
\ifdefined\remark\else\newtheorem*{remark}{Remark}\fi

%%%%%%%%%%%%%%%%%%%
% Text shortcuts  %
%%%%%%%%%%%%%%%%%%%
\renewcommand   {\th}{\textsuperscript{th}\xspace}
\newcommand     {\wrt}{\xspace{with respect to}\xspace}
\newcommand     {\Wrt}{\xspace{With respect to}\xspace}
%%% NOTE: \wlog is a logging command and can't be redefined
\newcommand     {\WLOG}{\xspace{without loss of generality}\xspace}
\newcommand     {\Wlog}{\xspace{Without loss of generality}\xspace}
\newcommand     {\str}[1]{\ifmmode{\mathstrut\text{#1}}\else\mbox{\strut#1}\fi}
\newcommand     {\TODO}[1]{\par \noindent \textbf{\underline{TODO:}}\space#1}
\newcommand     {\NOTE}[1]{\par \noindent \textbf{NOTE:}\space#1}
%%% Modifies enumerate* to use 1) 2) ...
\renewlist{enumerate}{enumerate}{1}
\setlist[enumerate]{label=\arabic*)}

\title{Computer Science and Mathematics Tutorial}
\author{Jacob White}
\date{\today}

\begin{document}
\maketitle

By default, the first line is indented (i.e. \textbackslash{parindent} is enabled); But, if parskip default in \href{run:./imports.tex}{imports.tex} is not overriden, this will not be indented.
This is on the same paragraph as the previous line. Newlines (\textbackslash \textbackslash) are necessary for regular lines.\\
The line thereafter is non-indented regardless, however.\\
This is a URL: \url{https://www.google.com}, and this is \href{https://www.google.com}{URL to Google}.\\
This is non-indented text on the next line.\\
\indent This is a manually-indented line.
\par\indent This is a manually-indented paragraph on its own line, with spacing on both ends if importing parskip.
\par\noindent This is a manually non-indented paragraph.
\par This paragraph has more space after it (newlines are unnecessary). \\
\par This is another paragraph.
This text line is part of the previous paragraph.\\
Put a \textbackslash\textbackslash\xspace or another \textbackslash{par} after a line in a multi-line paragraph to end the paragraph. 
\par\noindent However, you also need newlines before new environments even if it's a paragraph...\\

%\newpage
\begin{problem} 
    This is a problem. Lorem ipsum dolor sit amet, consectetur adipiscing elit, sed do eiusmod tempor incididunt ut labore et dolore magna aliqua. Ut enim ad minim veniam, quis nostrud exercitation ullamco laboris nisi ut aliquip ex ea commodo consequat. There is also some math in here: $a^2 + b^2 = c^2$.
\end{problem}
(An aside comment between problem and solution.)
\begin{solution}
    (solution environment starts with \textbackslash{par} to create indentation, and also ends page once done so each problem is on a separate line)
    \par This is a preliminary discussion about the solution.
    \begin{definition}
        This is a definition.
    \end{definition}
    \begin{definition*}
        This is an unnumbered definition.
    \end{definition*}
    \begin{definition}
        This is a second definition.
    \end{definition}
    \begin{assumption}
        This is an assumption for Problem \theproblem.
    \end{assumption}
    \begin{assumption*}
        This is an unnumbered assumption.
    \end{assumption*}
    \begin{claim}
        This is a claim for Problem \theproblem.
    \end{claim}
    \begin{claim*}
        This is an unnumbered claim.
    \end{claim*}
    \begin{fact}
        This is a (inherently unnumbered) fact.
    \end{fact}
    \begin{remark}
        This is a (inherently unnumbered) remark.
    \end{remark}
    \begin{observation}
        This is an (inherently unnumbered) observation.
    \end{observation}
    \begin{proposition}
        This is a (inherently unnumbered) proposition.
    \end{proposition}
    \begin{theorem}\label{babys-first-theorem}
        This is a theorem to assist in a solution for Problem \theproblem.
    \end{theorem}
    \begin{theoremproof}
        This is a proof for Theorem \ref{babys-first-theorem}.
    \end{theoremproof}
    \begin{corollary} \label{babys-first-corollary}
        This is a corollary for Theorem \ref{babys-first-theorem}.
    \end{corollary}
    \begin{corollaryproof}
        This is a proof for Corollary \ref{babys-first-corollary}.
    \end{corollaryproof}
    \begin{corollary} 
        This is another corollary for Theorem \ref{babys-first-theorem}.
    \end{corollary}
    \begin{lemma} \label{babys-first-lemma}
        This is a lemma for theorem \ref{babys-first-theorem}.
    \end{lemma}
    \begin{lemmaproof}
        This is a proof for Lemma \ref{babys-first-lemma}.
    \end{lemmaproof}
    
    \begin{theorem}[The Everything Theorem]
        This is a very special theorem.
    \end{theorem}
    \begin{proofsketch}
        This is an informal idea and/or intuition for the proof.
    \end{proofsketch}
    \begin{proof}
        This is a normal proof.
    \end{proof}
    \begin{corollary}[The Amazing Corollary]
        This is an extra special corollary.
    \end{corollary}
    
    \TODO{\hl{everything!}}
    \TODO{\st{done!}}
    \NOTE{This is a very useful note.}\\
%    \textrm{test}
%    \textmd{test}
    Italics: \textit{test}\\
    SC: \textsc{test}\\
    SL: \textsl{test}\\
%    \textssc{test}
%    \textsw{test}
    underline: \textul{test}\\
    underline2: \underline{test}\\
%    \textulc{test}
%    \textup{test}
    SF: \textsf{test}\\
    Algorithm font: \algoname{Algorithm-Name}\\
    String: \str{This is some string}\\
    
\end{solution}

\begin{problem}[The Difficult Problem; 100 pts]
    What is the answer to life, the universe, and everything?
\end{problem}
\begin{solution}
    \begin{theorem*}
        This is an unnumbered theorem.
    \end{theorem*}
    \begin{corollary*}
        This is an unnumbered corollary.
    \end{corollary*}
    \begin{lemma*}
        This is an unnumbered theorem.
    \end{lemma*}
    $$\str{bitstring} \in \bitset$$
    $$\powerset{\set{0,1}} = \set{\set{}, \set{0}, \set{1}, \set{0,1}}$$
    $$x \bigcdot y = z$$
    $$\str{octal} \in \lset{0}{7}^*$$
    $$\C = \cset{a + bi}{a,b \in \R}$$
    $$\Zmod{p} = \cset{n \in \N}{n < p}$$
    $\F$ is a generic field and $\G$ is a generic group.\\
    $$\nthroot{3}{27} = 3$$
    $$\abs{-3} = 3$$
    $$ A \iseq B$$
    $$\Pr{c \in C \given m \in M \land c = \algoname{Encrypt}(m) }$$
    If all instances in problem $A$ reduce to instances in problem $B$, then we say that $A \mapreducesto B$ and $B \mapreducesfrom A$.\\
    Different notations for set complement: $\complement{S}$, $\setcomplement{U}{S}$, $\minuscomplement{U}{S}$\\
    Language operations on strings of symbols in alphabet:
    \begin{itemize}
        \item Kleene star: $\kleenestar{A} = \cset{x_1 x_2 \ldots x_k}{k \geq 0 \land x_i \in A}$
        \item Concatenation: $\concat{A}{B} = \cset{xy}{x \in A \land y \in B}$
    \end{itemize}
    %%% Most of the math macros that can conceivably be used standalone in text can also be written without being enclosed in math environment delimiters as well.
    \par There are \approxtilde 7,000,000 humans in the world today. This statement is true (\true). (or maybe \false if you want to be precise...). 
    \par \abs{-1} is 1 and \nthroot{3}{27} is 3. \Pr{0=1} is 0. The set \set{1} contains only 1, and \cset{x}{x \in A} has a condition associated with it. This \bigcdot is a bullet. \complement{S} is the collection of elements in the "universe" set $U$ that aren't also in $S \subseteq U$.
    %%% Don't think too much about this, it's meaningless garbage. I was just too lazy to think of a proper math context to use these terms in.
    \par The world holds \WLOG to everything. \Wrt the universe, this should make sense. \Wlog, everything is \str{null} \wrt our perception.
    %%% This is all to show how \yields works
    \par We can define a Turing Machine configuration using strings $u,v \in \Sigma^*$, symbol $a \in \Sigma$, and state $q \in Q$ as $uqav$, where the head of the Turing Machine resides at the symbol $a$ in the input tape $uav$. Furthermore, we say that a configuration yields another, written e.g. $uqav \yields ubrv$, if there exists a transition function $\delta(q, a) = (r, b, R)$ that takes the Turing Machine from one configuration to the other. In this example, the Turing Machine reads $a$, writes $b$, transitions from state $q$ to $r$, and moves the head right by one position to the first symbol in $v$. This \blanksymbol is a blank symbol used in Turing Machine tapes but not in the alphabet.
    \par This is an empty string: \emptystring, and this is a trivial grammar: $S \derives \emptysymbol$. This is a string derivation using a grammar: $aS \derives abS \derives abc \equiv aS \derivesto abc$.
    \par This is a directed path from a to b: $a \pathto b$ and this is an undirected path: $a \undirpath b$.
    \par The maximum and minimum of 2 and 3 is $\max{2,3} = 3$ and $\min{2,3} = 2$.
\end{solution}

\end{document}
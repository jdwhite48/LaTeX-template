%%%%%%%%%%%%%%%%%%%%%%%%%%%%%%%%%%%%%%%%%%%%%%%%%%%%%%%%%%%%%%%%%%%%
%   Created By:     Jacob White
%   Email:          white570@purdue.edu
%
%%%%%%%%%%%%%%%%%%%%%%%%%%%%%%%%%%%%%%%%%%%%%%%%%%%%%%%%%%%%%%%%%%%%

\documentclass{article}
\usepackage[fontsize=12pt]{scrextend}   % specify 12pt font
\usepackage[margin=1in]{geometry}     % specify 1in margins

%%% Common template packages (must come before macros.tex)
%%% Formatting packages
\usepackage[utf8]{inputenc}
\usepackage{footnote}
\usepackage{xcolor}     % color names
\usepackage{graphicx}
\usepackage{url}        % URLs
\usepackage{hyperref}   % named URLs 
\usepackage{enumitem}   % better enumerations
\usepackage{multirow}   % multi-row math tables
\usepackage{multicol}   % multi-column pages and math tables
\usepackage{soul}       % strikethrough
\usepackage{parskip}    % Use decent \parskip. Also disables \parindent as default (must manually specify)

%%% Math packages
\usepackage{amsmath}
\usepackage{amssymb}    % \mathcal, \mathbb, etc.
\usepackage{amsthm}     % math theorem enumerations
\usepackage{amsfonts}
\usepackage{mathtools}  % \coloneqq, ...
\usepackage{bbm}
\usepackage{mathpartir}

%%% CS packages
\usepackage{ifthen}
\usepackage{xifthen}
\usepackage{listings}   % code

%%% Macros, environments, and other constructions
%%%%%%%%%%%%%%%%%%%%%%%%%%%%%%%%%%%%%%%%%%%%%%%%%%%%%%%%%%%%%%%%%%%%
%   Created By:     Jacob White
%   Email:          white570@purdue.edu
%   Last Updated:   27 August 2021
%
%   Sources of inspiration:
%       - Christina Garman's Practical & Applied Cryptography midterm template
%       - Reed Oei's math homework template (https://github.com/ReedOei/LaTeX)
%       - K A Gillow's University of Oxford, Mathematical Institute LaTeX Problem Sheet (https://www.maths.ox.ac.uk/members/it/faqs/latex/problem-sheets)
%       - Sources for specific commands may be referenced inline.
%
%%%%%%%%%%%%%%%%%%%%%%%%%%%%%%%%%%%%%%%%%%%%%%%%%%%%%%%%%%%%%%%%%%%%

\usepackage{amsmath}
\usepackage{amssymb}
\usepackage{amsthm}
\usepackage{footnote}

\usepackage{ifthen}
\usepackage{xifthen}
\usepackage{calc}

\usepackage{xspace}

%%% Change hyperref's colors
\hypersetup{            
     colorlinks=true,
     linkcolor=black,   % internal links (e.g. equation and theorem labels)
     urlcolor=cyan,     % \hyperref and \url
     filecolor=red,
     citecolor=black,
}

%%% Change color of footnote links to black
\renewcommand{\thefootnote}{\textcolor{black}{\arabic{footnote}}}

%%% From: https://en.wikibooks.org/wiki/LaTeX/Source_Code_Listings#Automating_file_inclusion
\newcommand{\includecode}[2][c]{\lstinputlisting[language=#1, caption=#2, basicstyle=\normalsize\ttfamily]{#2}}

%%% Better paragraph indents
%\setlength{\parskip}{0em}       % parskip import does this already -- adds spacing after paragraph "by about the length of a lowercase 'x'"
\AtBeginDocument{\setlength{\parindent}{15pt}}     % Re-enable \parindent as default (equiv. spacing for 11pt) even if parskip imported

%%% Math macros
\newcommand     {\mathify}[1]{\ifmmode{#1}\else\mbox{$#1$}\fi}

\newcommand     {\group}{\mathify{\mathbb{G}}\xspace}
\newcommand     {\G}{\group}
\newcommand     {\nat}{\mathify{\mathbb{N}}\xspace}
\newcommand     {\N}{\nat}
\newcommand     {\integer}{\mathify{\mathbb{Z}}\xspace}
\newcommand     {\integers}{\integer}
\newcommand     {\Z}{\integer}
\newcommand     {\Zmod}[1]{\mathify{\Z_{#1}}\xspace}
\newcommand     {\rat}{\mathify{\mathbb{Q}}\xspace}
\newcommand     {\rational}{\rat}
\newcommand     {\rationals}{\rat}
\newcommand     {\Q}{\rat}
\newcommand     {\real}{\mathify{\mathbb{R}}\xspace}
\newcommand     {\reals}{\real}
\newcommand     {\R}{\real}
\newcommand     {\complex}{\mathify{\mathbb{C}}\xspace}
\newcommand     {\C}{\complex}
\newcommand     {\field}{\mathify{\mathbb{F}}\xspace}
\newcommand     {\F}{\field}
%\newcommand     {\ma}{\mathify{\mathcal{A}} }
%\newcommand     {\mb}{\mathify{\mathcal{B}} }
\newcommand     {\powerset}[1]{\mathify{\mathcal{P}(#1)}\xspace}
% \renewcommand   {\implies}{ \longrightarrow }
%%% For whatever reason \mathify doesn't add spaces around sides in math mode
\newcommand     {\bigcdot}{\ensuremath{\bullet}\xspace}
\newcommand     {\qeq}{\stackrel{?}{=}}
\newcommand     {\approxtilde}{\raise.17ex\hbox{\mathify{\scriptstyle\sim}}}
\newcommand     {\nthroot}[2]{\mathify{\sqrt[\leftroot{-1}\uproot{1}\scriptstyle #1]{#2}}}
\newcommand     {\abs}[1]{\mathify{\left| #1 \right|}}
\renewcommand   {\Pr}[1]{\mathify{\mbox{Pr}\left[#1\right]}}
\newcommand     {\set}[1]{\mathify{\left\{ #1 \right\}}\xspace}
%%% Change to colon if you prefer that conditional notation over a vertical bar (\mid)
%\newcommand {\suchthat}{\xspace:\xspace}
\newcommand     {\suchthat}{\xspace\mid\xspace}
\newcommand     {\given}{\suchthat}
%%% Prevents overloading strikethrough macro if the document imports soul package
\ifdefined\st\else\newcommand{\st}{\suchthat}\fi
\newcommand     {\cset}[2]{\set{#1 \suchthat #2}}  % a conditional notation to define sets
\newcommand     {\lset}[2]{\set{#1,\ldots,#2}} % set {from,...,to}
\renewcommand   {\complement}[1]{\mathify{\overline{#1}}}
\newcommand     {\setcomplement}[2]{\mathify{#1\setminus{#2}}}
\newcommand     {\minuscomplement}[2]{\mathify{#1-#2}}

%%% CS Macros
\newcommand     {\true}{\textbf{\textsf{T}}\xspace}
\newcommand     {\false}{\textbf{\textsf{F}}\xspace}
\newcommand     {\bitset}{\mathify{\{0,1\}^*}\xspace}
\newcommand     {\mapreducesto}{\leq_m}
\newcommand     {\mapreducesfrom}{\geq_m}
\newcommand     {\yields}{\rightarrow_M}
%%% From: https://tex.stackexchange.com/questions/41357/replace-cmtt10-char32-visible-cup-space-with-something-more-gentle/41445#41445
\newcommand     {\blanksymbol}{%
  \ooalign{%
    \relax\cr%
    \noalign{\vskip.2ex}%
    \textvisiblespace\cr%
    \noalign{\vskip-.2ex}%
    \hphantom{~}\cr%   Width of a space character
    \vphantom{Xg}\cr%  Height of X, depth of g
  }\xspace%
}
\newcommand   {\concat}[2]{#1 \circ #2}
\newcommand   {\kleenestar}[1]{{#1}^*}

%%% Various theorem and problem and proof environments
\newcounter{problem}
\renewcommand{\theproblem}{\arabic{problem}}    
\newenvironment{problem}[1][] { % Optionally include descriptor and make following text italics by default
    \stepcounter{problem} \noindent \underline{\textbf{Problem \theproblem}}:
    \ifx\newenvironment#1\newenvironment\else\space(#1)\fi \space\itshape 
} {\vspace{1em}}

\newenvironment{solution} {\par} {\newpage}
%\renewenvironment{proof} {\vspace{0.5em} \noindent \underline{\textbf{Pf}}:} {\hfill \qedsymbol \vspace{0.5em}}      % Default proof, but with _*Pf:*_ instead
\newenvironment{theoremproof} {\vspace{0.5em} \noindent \textit{Proof \thetheorem.}} {\hfill \qedsymbol \vspace{0.5em}}      % Default proof, numbered by theorem
\newenvironment{corollaryproof} {\vspace{0.5em} \noindent \textit{Proof \thecorollary.}} {\hfill \qedsymbol \vspace{0.5em}}      % Default proof, numbered by corollary
\newenvironment{lemmaproof} {\vspace{0.5em} \noindent \textit{Proof \thelemma.}} {\hfill \qedsymbol \vspace{0.5em}}      % Default proof, numbered by lemma
\newenvironment{proofsketch} {\vspace{0.5em} \noindent \textit{Proof (Sketch).}} {\vspace{0.5em}}      % Default proof, but as an informal description

\ifdefined\definition\else\newtheorem{definition}{Definition}\fi
\newtheorem*{definition*}{Definition}
\ifdefined\theorem\else\newtheorem{theorem}{Theorem}[problem]\fi
\newtheorem*{theorem*}{Theorem}
%%% TODO: How do I want to number assumptions? Is a separate counter sufficient?
\ifdefined\assumption\else\newtheorem{assumption}{Assumption}[problem]\fi
\newtheorem*{assumption*}{Assumption}
\ifdefined\corollary\else\newtheorem{corollary}{Corollary}[theorem]\fi
\newtheorem*{corollary*}{Corollary}
\ifdefined\lemma\else\newtheorem{lemma}{Lemma}[theorem]\fi
\newtheorem*{lemma*}{Lemma}
\ifdefined\observation\else\newtheorem*{observation}{Observation}\fi
\ifdefined\proposition\else\newtheorem*{proposition}{Proposition}\fi
\ifdefined\claim\else\newtheorem*{claim}{Claim}\fi
\ifdefined\fact\else\newtheorem*{fact}{Fact}\fi
\ifdefined\remark\else\newtheorem*{remark}{Remark}\fi

%%% Text shortcuts
\newcommand     {\wrt}{\xspace{with respect to}\xspace}
\newcommand     {\Wrt}{\xspace{With respect to}\xspace}
%%% NOTE: \wlog is a logging command and can't be redefined
\newcommand     {\WLOG}{\xspace{without loss of generality}\xspace}
\newcommand     {\Wlog}{\xspace{Without loss of generality}\xspace}
\newcommand     {\str}[1]{\ifmmode{\mathstrut\text{#1}}\else\mbox{\strut#1}\fi}
\newcommand     {\algoname}[1]{\textsf{#1}}
\newcommand     {\TODO}[1]{\par \noindent \textbf{\underline{TODO:}}\space#1}
\newcommand     {\NOTE}[1]{\par \noindent \textbf{NOTE:}\space#1}

\title{Computer Science and Mathematics Tutorial}
\author{Jacob White}
\date{\today}

\begin{document}
\maketitle
%\jw{Author notes!}
By default, the first line is indented (i.e. \textbackslash{parindent} is enabled); But, if parskip default in \href{run:./imports.tex}{imports.tex} is not overriden, this will not be indented.
This is on the same paragraph as the previous line. Newlines (\textbackslash \textbackslash) are necessary for regular lines.\\
The line thereafter is non-indented regardless, however.\\
This is a URL: \url{https://www.google.com}, and this is \href{https://www.google.com}{URL to Google}.\\
This is non-indented text on the next line.\\
\indent This is a manually-indented line.
\par\forceindent This is a manually-indented paragraph on its own line, with spacing on both ends if importing parskip.
\par\noindent This is a manually non-indented paragraph.
\par This paragraph has more space after it (newlines are unnecessary). \\
\par This is another paragraph.
This text line is part of the previous paragraph.\\
Put a \textbackslash\textbackslash\xspace or another \textbackslash{par} after a line in a multi-line paragraph to end the paragraph. 
\par\noindent However, you also need newlines before new environments even if it's a paragraph...\\

%\newpage
\begin{problem} 
    This is a problem. Lorem ipsum dolor sit amet, consectetur adipiscing elit, sed do eiusmod tempor incididunt ut labore et dolore magna aliqua. Ut enim ad minim veniam, quis nostrud exercitation ullamco laboris nisi ut aliquip ex ea commodo consequat. There is also some math in here: $a^2 + b^2 = c^2$.
\end{problem}
(An aside comment between problem and solution.)
\begin{solution}
    (solution environment starts with \textbackslash{par} to create indentation, and also ends page once done so each problem is on a separate line)
    \par This is a preliminary discussion about the solution.
    \begin{definition}
        This is a definition.
    \end{definition}
    \begin{definition*}
        This is an unnumbered definition.
    \end{definition*}
    \begin{definition}
        This is a second definition.
    \end{definition}
    \begin{assumption}
        This is an assumption for Problem \theproblem.
    \end{assumption}
    \begin{assumption*}
        This is an unnumbered assumption.
    \end{assumption*}
    \begin{claim}
        This is a claim for Problem \theproblem.
    \end{claim}
    \begin{claim*}
        This is an unnumbered claim.
    \end{claim*}
    \begin{fact}
        This is a (inherently unnumbered) fact.
    \end{fact}
    \begin{remark}
        This is a (inherently unnumbered) remark.
    \end{remark}
    \begin{observation}
        This is an (inherently unnumbered) observation.
    \end{observation}
    \begin{proposition}
        This is a (inherently unnumbered) proposition.
    \end{proposition}
    \begin{theorem}\label{babys-first-theorem}
        This is a theorem to assist in a solution for Problem \theproblem.
    \end{theorem}
    \begin{theoremproof}
        This is a proof for Theorem \ref{babys-first-theorem}.
    \end{theoremproof}
    \begin{corollary} \label{babys-first-corollary}
        This is a corollary for Theorem \ref{babys-first-theorem}.
    \end{corollary}
    \begin{corollaryproof}
        This is a proof for Corollary \ref{babys-first-corollary}.
    \end{corollaryproof}
    \begin{corollary} 
        This is another corollary for Theorem \ref{babys-first-theorem}.
    \end{corollary}
    \begin{lemma} \label{babys-first-lemma}
        This is a lemma for theorem \ref{babys-first-theorem}.
    \end{lemma}
    \begin{lemmaproof}
        This is a proof for Lemma \ref{babys-first-lemma}.
    \end{lemmaproof}
    
    \begin{theorem}[The Everything Theorem]
        This is a very special theorem.
    \end{theorem}
    \begin{proofsketch}
        This is an informal idea and/or intuition for the proof.
    \end{proofsketch}
    \begin{proof}
        This is a normal proof.
    \end{proof}
    \begin{corollary}[The Amazing Corollary]
        This is an extra special corollary.
    \end{corollary}
    
    \TODO{\hl{everything!}}
%    \TODO{\st{done!}}
    \NOTE{This is a very useful note.}\\
%    \textrm{test}
%    \textmd{test}
    Italics: \textit{test}\\
    SC: \textsc{test}\\
    SL: \textsl{test}\\
%    \textssc{test}
%    \textsw{test}
    underline: \textul{test}\\
    underline2: \underline{test}\\
%    \textulc{test}
%    \textup{test}
    SF: \textsf{test}\\
    Algorithm font: \alg{Algorithm-Name}\\
    String: \str{This is some string}\\
    
\end{solution}

\begin{problem}[The Difficult Problem; 100 pts]
    What is the answer to life, the universe, and everything?
\end{problem}
\begin{solution}
    \begin{theorem*}
        This is an unnumbered theorem.
    \end{theorem*}
    \begin{corollary*}
        This is an unnumbered corollary.
    \end{corollary*}
    \begin{lemma*}
        This is an unnumbered theorem.
    \end{lemma*}

    $$\str{boolean} \in \boolset$$
    $$\str{arbitrary bitstring} \in \bitsset$$
    $$n\str{-bit bitstring} \in \bitsset[n]$$
    $$\powerset{\set{0,1}} = \set{\set{}, \set{0}, \set{1}, \set{0,1}}$$
    $$x \bigcdot y = z$$
    $$\str{octal} \in \lset{0}{7}^*$$
    $$\complex = \cset{a + bi}{a,b \in \reals}$$
    $$\Zmod{p} = \cset{n \in \N}{n < p}$$
    $\field$ is a generic field and $\group$ is a generic group.\\
    $$\nthroot{3}{27} = 3$$
    $$\abs{-3} = 3$$
    $$ A \iseq B$$
    $$\Prob{[c \in C \given m \in M \land c = \alg{Encrypt}(m)]}$$

    If all instances in problem $A$ reduce to instances in problem $B$, then we say that $A \mapreducesto B$. If it is also true that $B \mapreducesfrom A$, then these problems are equivalent; that is, $A \mapequiv B$. Similar for polynomial-time reductions $A \polyreducesto B$ and log-space reductions $A \logspacereducesto B$. \\

    Language operations on strings of symbols in alphabet:
    \begin{enumerate}
        \item Kleene star: $\kleenestar{A} = \cset{x_1 x_2 \ldots x_k}{k \geq 0 \land x_i \in A}$
        \item Concatenation: $A \langconcat B = \cset{xy}{x \in A \land y \in B}$. String concatenation can also be denoted by $x \strconcat y$.
    \end{enumerate}

    \par \secparam is the usual notation for the security parameter, and is often written in unary form as $1^\secparam$. The set of booleans is \boolset, alternatively written as \set{0,1}. In finite field arithmetic, the additive identity is often denoted by the "point at infinity" \inftypt. This is a common reference string \crs and this is a relation \rel{}.

    %%% Most of the math macros that can conceivably be used standalone in text can also be written without being enclosed in math environment delimiters as well.
    \par There are \approx 7,000,000 humans in the world today. This statement is true (\true). (or maybe \false depending on how precise you want to be...). I am \inverse{7{,}000{,}000} of the world's population. $\approx\half$ of the world's population is male.
    \par \abs{-1} is 1 and \nthroot{3}{27} is 3. $\Prob{[0=1]}$ is 0. The expected value is $\Exp_{X}[1] = 1$ and the probability density is $\Prob_{X}(x)$ with distribution variance $\Var(X)$ and covariance $\Covar(X)$. The set \set{1} contains only 1, and \cset{x}{x \in A} has a condition associated with it. 5 is the 6\th number in \N. This $A \bigcdot B$ is a bullet. \complement{S} is the collection of elements in the "universe" set $U$ that aren't also in $S \subseteq U$. Different notations for set complement: $\complement{S}$, $U \setminus S$, $U - S$\\
    \par This is a transpose of a matrix $A$: $\transpose{A}$. This is a complex conjugate transpose: $\cctranspose{A}$.
    %%% Don't think too much about this, it's meaningless garbage. I was just too lazy to think of a proper math context to use these terms in.
    \par The world holds \WLOG to everything. \Wrt the universe, this should make sense. \Wlog, everything is \str{null} \wrt our perception.

    %%% This is all to show how \yields works
    \par We can define a Turing Machine configuration using strings $u,v \in \Sigma^*$, symbol $a \in \Sigma$, and state $q \in Q$ as $uqav$, where the head of the Turing Machine resides at the symbol $a$ in the input tape $uav$. Furthermore, we say that a configuration yields another, written e.g. $uqav \yields ubrv$, if there exists a transition function $\delta(q, a) = (r, b, R)$ that takes the Turing Machine from one configuration to the other. In this example, the Turing Machine reads $a$, writes $b$, transitions from state $q$ to $r$, and moves the head right by one position to the first symbol in $v$. This \blanksymbol is a blank symbol used in Turing Machine tapes but not in the alphabet.
    \par This is an empty string: \emptystring, and this is a trivial grammar: $S \derives \emptysymbol$. This is a string derivation using a grammar: $aS \derives abS \derives abc \equiv aS \derivesto abc$. This is a partial reduction from $A$ to $B$: $A \partreducesto B$. This is an evaluation of expression $e$ to value $v$ in some state \state: $\state, e \evaluatesto v$. This is heta, representing the state of the heap: \heap. "$\Gamma \assumedby A$" reads as "assuming proposition(s) in $\Gamma$ are true, $A$ is true".
    This is an inference rule:
    
    \begin{mathpar}
        \inferrule* [rightstyle=\sf, Right=EWrite]
            { (\heap, \state, \locmap), e_1 \evaluatesto v_1 \\ (\heap, \state, \locmap), e_2 \evaluatesto v_2 }
            {  (\heap, \state, \locmap), *[e_1] \gets e_2 \evaluatesto ([v_1 \mapsto v_2]\, \heap, \state, \locmap) }
    \end{mathpar}

    This is a proof tree:

    \begin{mathpar}
        \inferrule* [rightstyle=\sf, Right=EAdd] {
            \inferrule* [rightstyle=\sf, Right=ESub, rightskip=-1em] {
                \inferrule* [rightstyle=\sf, Right=ENum, rightskip=-1em] { } { \state, 5 \evaluatesto 5 } \\
                \inferrule* [rightstyle=\sf, Right=ENum, leftskip=-1em] { } { \state, 2 \evaluatesto 2 }
            }
            { \state, 5 - 2 \evaluatesto 5 -_\nat 2 } \\
            \inferrule* [rightstyle=\sf, Right=ENum, leftskip=-1em] { } { \state, 3 \evaluatesto 3 }
        }
        { \sigma, 5 - 2 + 3 \evaluatesto 5 -_\nat 2 +_\nat 3 }
    \end{mathpar}
    
    This is syntax:
    \begin{align*}
        A &\syntaxeq N\\
        &\mid A + A\\
        &\mid A - A\\
        &\mid X\\
        X &\in \mathsf{Id}\\
        N &\in \nat
    \end{align*}
    

    \par This is a directed path from $a$ to $b$: $a \pathto b$ and this is an undirected path: $a \undirpathto b$. This is a walk from $a$ to $b$: $a \walkto b$.

    Kronecker delta \kron{i}{j}! Sample space \samplespace b! Disjoint union $A \cupdisjoint B$ means that $A \cup B \land A \cap B = \emptyset$. This is a vector \vec{A} (assumed $n$-by-1 column vector by default). This is the number of dimensions of a finite-dimensional vector space $V$: $\dim V$. For a sum of vector spaces, let the direct sum be $U \directsum V \gets \cset{\vec{u} + \vec{v}}{\vec{u} \in U, \vec{v} \in V, U \cap V = \set{0}}$. $A \iso B$ denotes that $A$ and $B$ are isomorphic. The nullspace/kernel of a linear map/transformation $T$ is $\nullspace T$. The norm on a real vector space $V$ is $\norm{\ignored} : V \to \reals$ (e.g., \norm{\vec{a}} and \norm{\vec{b}}; generalizes length). Any inner product on $V$ induces a norm on $V$: $$\norm{\vec{x}} = \sqrt{\innerprod{\vec{x}}{\vec{x}}}$$ $\Max \set{1,3,5} = 5$ and  $\min\set{1,3,5} = 1$. Two vectors \vec{x} and \vec{y} are orthogonal if $\innerprod{\vec{x}}{\vec{y}} = 0$: i.e., $\vec{x} \orthogonal \vec{y}$. We define the orthogonal complement of $S$, denoted $\orthotrans{S}$, as the set of all vectors in $V$ orthogonal to every element in $S$; $V = S \directsum \orthotrans{S}$ for any subspace $S \subseteq V$. A nonzero vector $\vec{x} \in \R^n$ is an eigenvector of matrix \mat{A} corresponding to eigenvalue \eigenval if $\mat{A} \vec{x} = \eigenval \vec{x}$. A trace of a square matrix is the sum of its diagonal entries AND the sum of its eigenvalues (repeated by multiplicity): $\trace \mat{A} = \sum_{i=1}^n A_{i\,i} = \sum_i \eigenval_i(\mat{A})$. The determinant is the product of a matrix \mat{A}'s eigenvalues: $\det \mat{A} = \prod_i \eigenval_i(\mat{A})$. A symmetric matric $\mat{A}$ (i.e., where $\mat{A} = \transpose{\mat{A}}$) is positive semi-definite if, for all $\vec{x} \in \R^n, \transpose{\vec{x}} \mat{A} \vec{x} \geq 0$; this is often denoted $\mat{A} \succeq 0$, and is true iff all of its eigenvalues are non-negitive.

    Every matrix $\mat{A} \in \R^{m \times n}$ has a singular value decomposition (SVD) $\mat{A} = \mat{U} \mat{\Sigma} \transpose{\mat{V}}$ where $\mat{U \in \R^{m \times m}}$ and $\mat{V} \in \R^{n \times n}$ are orthogonal matrices and $\mat{\Sigma} \in \R^{m \times n}$ is a diagonal matrix of its singular values (denoted $\sv_i$): by convention, $\sv_1 \geq \sv_2 \geq \cdots \geq \sv_r > \sv_{r+1} = \cdots = \sv_{\min(m, n)} = 0$ where $r = \rank \mat{A}$. These SVD factors provide eigendecompositions for $\transpose{\mat{A}}\mat{A} = \mat{V} \transpose{\mat{\Sigma}} \mat{\Sigma} \transpose{\mat{V}}$ and $\mat{A}\transpose{\mat{A}} = \mat{U} \mat{\Sigma} \transpose{\mat{\Sigma}} \transpose{\mat{U}}$: columns of \mat{V} are eigenvectors of $\transpose{\mat{A}}\mat{A}$, columns of \mat{U} are eigenvectors of $\transpose{\mat{A}}\mat{A}$.

    SVD is useful for many computations, including low-rank approximation: i.e., given some matrix, find another matrix with the same dimensions but lower rank such that the two matrices are close according to some norm.

    \begin{theorem} [Eckart-Young-Mirsky theorem]
        Let \norm{\ignored} be a unitary invariant matrix norm. Suppose $\mat{A} \in \R^{m \times n}$, where $m \geq n$, has SVD $\mat{A} = \sum_{i=1}^n \sv_i \vec{u}_i \transpose{\vec{v}_i}$. Then the best rank-$k$ approximation to \mat{A}, where $k \leq \rank(\mat{A})$, is given by:
        $$\mat{A}_k = \sum_{i=1}^k \sv_i \vec{u}_i \transpose{\vec{v}_i}R$$
        wherein $\norm{\mat{A} - \mat{A}_k} \leq \norm{\mat{A} - \hat{\mat{A}}}$ for any $\hat{\mat{A}} \in \R^{m \times n}$ with $\rank(\hat{\mat{A}}) \leq k$.
    \end{theorem}

    A (Moore-Penrose) pseudoinverse $\pseudoinv{\mat{A}} \in \R^{n \times m}$ exists for ANY matrix $\mat{A} \in \R^{m \times n}$, with:
    \begin{enumerate*}
        \item $\mat{A} \pseudoinv{\mat{A}} \mat{A} = \mat{A}$
        \item $\pseudoinv{\mat{A}} \mat{A} \pseudoinv{\mat{A}} = \pseudoinv{\mat{A}}$
        \item $\mat{A} \pseudoinv{\mat{A}}$ and $\pseudoinv{\mat{A}} \mat{A}$ is symmetric.
    \end{enumerate*} If \mat{A} is invertibble, then $\pseudoinv{\mat{A}} = \multinv{\mat{A}}$; more generally, if $\mat{A} = \mat{U} \mat{\Sigma} \transpose{\mat{V}}$, then $\pseudoinv{\mat{A}} = \mat{V} \mat{\Sigma}^\dagger \transpose{\mat{U}}$

    \begin{theorem}[Fundamental Theorem of Algebra]
    If $\mat{A} \in \R^{m \times n}$, then:
    \begin{enumerate}
        \item $\nullspace(\mat{A}) = \orthotranspose{\range(\transpose{\mat{A}})}$
        \item $\nullspace(\mat{A}) \directsum \range(\transpose{\mat{A}}) = \R^n$
        \item (Rank-nullity theorem) $\underbrace{\dim \range(\mat{A})}_{\rank(\mat{A})} + \dim \nullspace(\mat{A}) = n$
        \item If $\mat{A} = \mat{U} \mat{\Sigma} \transpose{\mat{V}}$ is the SVD of $\mat{A}$, then the columns of \mat{U} and \mat{V} form orthonormal bases for the four "fundamental subspaces of \mat{A}.
        \begin{enumerate*}
        \item $\range(\mat{A})$: The first $r$ columns of \mat{U}
        \item $\range(\transpose{\mat{A}})$: The first $r$ columns of \mat{V}
        \item $\nullspace(\transpose{\mat{A}})$: The last $m-r$ columns of \mat{U}
        \item $\nullspace(\mat{A})$: The last $n-r$ columns of \mat{V}
        \end{enumerate*}
    \end{enumerate}
    \end{theorem}

    Recall the sample space \samplespace as a fixed set of outcomes $\sample \in \samplespace$. Let the event space $\eventspace \subseteq \powerset{\samplespace}$ represent a set of events $\event \in \eventspace$, i.e. each event represents a set of outcomes for a particular experiment. We can define a probability measure $\Prob : \eventspace \to \set{0,1}$ which satisfies \begin{enumerate*}
        \item $\Prob(\samplespace) = 1$
        \item \textbf{Countable additivity:} for any countable collection of disjoint sets $\set{A_i} \subseteq \eventspace$, $\Prob\left(\bigcup_i A_i \right) = \sum_i \Prob(A_i)$.
    \end{enumerate*}
    \probspace is a probability space. By Bayes' rule, $$\Prob(A \given B) \propto \Prob(A) \Prob(B \given A)$$ with prior $\Prob(A)$, posterior $\Prob(A \given B)$, and likelihood $\Prob(B \given A)$.

    A random value is an uncertain quantity with an associated probability distribution over its assumed values, e.g. a measurable function $X : \samplespace \to \R$ (or codomain into any other measurable space).
    
    A cumulative distribution function (CDF) gives the probability that a random value is at most a certain value: $F(x) = \Prob(X \leq x)$, with $\Prob(a < X \leq b$ = $F(b) - F(a)$. A discrete random value is specified by its probability mass function (PMF) $p : X(\samplespace) \to [0,1]$ which satisfies $\sum_{x \in X(\samplespace)} p(x) = 1$, wherein $\Prob(X = x) = p(x)$. Note that there's a common abuse of notation, wherein $p(X)$ is a probability distribution of $x$ and $p(x; \theta)$ is an evaluation at the value $x \in X(\samplespace)$, parameterized by $\theta$ (if distinct). A continuous random variable has a corresponding probability density function (PDF) which satisfies $\int_{-\infty}^{\infty} p(x) dx = 1$ (NOTE: it is possible in continuous values to have $p(x) > 1$; the function $p$ does NOT itself give probabilities).

    An expected value $\Exp[X]$ captures the "average" or "center" (mean) value of $X$, and is defined as $\Exp[X] = \sum_{x \in X(\samplespace)} x \cdot p(x)$ (discrete) or $\Exp[X] = \int_{-\infty}^{\infty} x p(x) dx$ (continuous). Importantly, the expected value is linear (even if $X$ is not independent!), i.e.: $\Exp\left[\sum_{i=1}^n \alpha_i X_i + \beta \right] = \sum_{i=1}^n \alpha_i \Exp[X_i] + \beta$. Furthermore, if $X$ is independent, the product rule also holes: $\Exp\left[\prod_{i=1}^n X_i\right] = \prod_{i=1}^n \Exp[X_i]$. Expectation applies entry-wise in a vector.

    Variance is a measure of the "spread" around the center of a distribution, defined by $\Var(X) = \Exp\left[(X-\Exp[X])^2\right] = \Exp[X^2] - \Exp[X]^2$. It is not linear, but: $\Var(\alpha X + \beta) = \alpha^2 \Var(X)$ (multiplicative squared, additive dissapear). Furthermore, if all $X_1, \ldots, X_n$ are uncorrelated, then $\Var(X_1 + \cdots + X_n) = \Var(X_1) + \cdots + \Var(X_n)$. However, if we want the same units as the random variable $X$ itself, use its standard deviation $\sd = \sqrt{\Var(X)}$.

    Covariance is a measure of the linear relationship between two variables, say $X$ and $Y$: $\Covar(X,Y) = \Exp[(X - \Exp[X])(Y - \Exp[Y])]$ where the outer expectation is over the joint distribution of $X$ and $Y$. By linearity of expectation: $\Covar(X, Y) = \Exp[XY] - \Exp[X]\Exp[Y]$. Covariance is also bilinear:
    \begin{align*}
    \Covar(\alpha X + \beta Y, Z) &= \alpha\Covar(X, Z) + \beta\Covar(Y,Z)\\
    \Covar(X, \alpha Y + \beta Z) &= \alpha\Covar(X, Y) + \beta\Covar(Y,Z)
    \end{align*}

    Correlation is a normalization of covariance, also measuring the linear relationship between variables but ranging between $[-1, 1]$:
    $$\corr(X,Y) = \frac{\Covar(X,Y)}{\sqrt{\Var(X)\Var(Y)}}$$. $\Covar(X,Y) = 0 \implies \corr(X,Y) = 0$, i.e. that $X$ and $Y$ are uncorrelated. If two variables are independent then they are uncorrelated, but the converse is not necessarily true. Furthermore, covariance becomes a matrix \covarmat with $\covarmat_{ij} = \Covar(X_i, X_j)$. Since covariance is symmetric ($\Covar(X_i, X_j) = \Covar(X_j, X_i)$), this makes the matrix itself symmetric (and is also positive semi-definite). The inverse of a covariance matrix $\inverse{\covarmat}$ is a precision matrix.

    Maximum likeliohood estimation (MLE) is a common way to fit parameters, "explaining" the data by maximizing probability/density of the data as a function of the parameters, i.e. $\hat{\theta}_{MLE} = \operatorname{arg max}_\theta \likelihood(\theta)$, where $\likelihood(\theta) = p(x_1, \ldots, x_n; \theta)$ (against the observations); if $X_1, \ldots, X_n$ are iid, then $\ldots = \prod_{i=1}^n p(x_i; \theta)$. We can convert this to logarithms for convenience (since positive probabilities and monotonically increasing $\log$):
    $$\log \likelihood(\theta) = \sum_{\i=1}^n \log p(x_i; \theta)$$
    Then, solve maximum likelihood estimator for $\theta$.

\end{solution}

\end{document}
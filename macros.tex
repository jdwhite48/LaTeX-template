%%%%%%%%%%%%%%%%%%%%%%%%%%%%%%%%%%%%%%%%%%%%%%%%%%%%%%%%%%%%%%%%%%%%
%   Created By:     Jacob White
%   Email:          white570@purdue.edu
%
%   Sources of inspiration:
%       - Christina Garman's Practical & Applied Cryptography midterm template
%       - Reed Oei's math homework template (https://github.com/ReedOei/LaTeX)
%       - K A Gillow's University of Oxford, Mathematical Institute LaTeX Problem Sheet (https://www.maths.ox.ac.uk/members/it/faqs/latex/problem-sheets)
%       - Sources for specific commands may be referenced inline.
%
%%%%%%%%%%%%%%%%%%%%%%%%%%%%%%%%%%%%%%%%%%%%%%%%%%%%%%%%%%%%%%%%%%%%

%%% Change hyperref's colors
\hypersetup{            
  colorlinks=true,
  linkcolor=black,   % internal links (e.g. equation and theorem labels)
  urlcolor=cyan,     % \hyperref and \url
  filecolor=red,
  citecolor=black,
}

%%% Author notes
%\newcommand{\jw}[1]{\dtcolornote[JW]{cyan}{#1}}

%%% Change color of footnote links to black
%\renewcommand{\thefootnote}{\textcolor{black}{\arabic{footnote}}}

\newcommand\mycommfont[1]{\footnotesize\ttfamily\textcolor{blue}{#1}}
\SetCommentSty{mycommfont}
%%% From: https://en.wikibooks.org/wiki/LaTeX/Source_Code_Listings#Automating_file_inclusion
\newcommand{\includecode}[3][c]{\lstinputlisting[
  language=#1,
  caption=#2,
  basicstyle=\normalsize\ttfamily,
  breaklines=true,
  postbreak=\mbox{\textcolor{red}{$\hookrightarrow$}\space}
]{#3}}

%%% Better paragraph indents
%\setlength{\parskip}{0em}       % parskip import does this already -- adds spacing after paragraph "by about the length of a lowercase 'x'"
%\AtBeginDocument{\setlength{\parindent}{15pt}}     % Re-enable \parindent as default (equiv. spacing for 11pt) even if parskip imported
\newcommand{\forceindent}{\leavevmode{\parindent=15pt\indent}}

\newcommand{\cmark}{\ding{51}}%
\newcommand{\xmark}{\ding{55}}%

%%%%%%%%%%%%%%%%%%%
% Math macros     %
%%%%%%%%%%%%%%%%%%%

%%% General Arithmetic
\newcommand{\nthroot}[2]{\ensuremath{\sqrt[\leftroot{-1}\uproot{1}\scriptstyle #1]{#2}}}
\newcommand{\abs}[1]{\ensuremath{\left| #1 \right|}\xspace}
\newcommand{\norm}[1]{\ensuremath{\left\lVert #1 \right\rVert}\xspace}
\newcommand{\innerprod}[2]{\ensuremath{\left\langle #1,#2 \right\rangle}\xspace}
\newcommand{\fracinv}[1]{\ensuremath{\frac{1}{#1}}\xspace}
\newcommand{\half}{\fracinv{2}}
\newcommand{\third}{\fracinv{3}}
\newcommand{\fourth}{\fracinv{4}}
\newcommand{\addinv}[1]{\ensuremath{-{#1}}\xspace}
\newcommand{\pseudoinv}[1]{\ensuremath{{#1}^{\!+}}\xspace}
\newcommand{\multinv}[1]{\ensuremath{{#1}^{-1}}\xspace}
\newcommand{\inverse}[1]{\multinv{#1}}
\newcommand{\kron}[2]{\ensuremath{\delta_{#1\,#2}}\xspace}
%\newcommand{\ceil}[1]{\left\lceil #1 \right\rceil}
%\newcommand{\floor}[1]{\left\lfloor #1 \right\rfloor}


%%% Linear Algebra
% NOTE: Kernel and nullspace are equivalent in linear algebra, but not in other domains (e.g., machine learning). If both Math and CS are relevant, use nullspace.
%\DeclareMathOperator{\Ker}{Ker}
\DeclareMathOperator{\nullspace}{null}
%\DeclareMathOperator{\nullspace}{\Ker}
\DeclareMathOperator{\trace}{tr}
\DeclareMathOperator{\rank}{rank}
\DeclareMathOperator{\range}{range}
\newcommand{\directsum}{\mathbin{\oplus}}
\newcommand{\cupdot}{\mathbin{\mathaccent\cdot\cup}}
%\newcommand{\cupdisjoint}{\sqcup}
\newcommand{\cupdisjoint}{\cupdot}
\newcommand{\orthogonal}{\mathbin{\bot}}
\newcommand{\transpose}[1]{\ensuremath{{#1}^\top}\xspace}
\newcommand{\trans}[1]{\transpose{#1}}
\newcommand{\orthotranspose}[1]{\ensuremath{{#1}^\bot}\xspace}
\newcommand{\orthotrans}[1]{\orthotranspose{#1}}
%%% Complex Conjugate transpose -- ^* or ^H for Linear Algebra, dagger for quantum mechanics
%\newcommand{\cctranspose}[1]{\ensuremath{{#1}^*}\xspace}
%\newcommand{\cctranspose}[1]{\ensuremath{{#1}^H}\xspace}
\newcommand{\cctranspose}[1]{\ensuremath{{#1}^\dagger}\xspace}
\newcommand{\cctrans}[1]{\cctranspose{#1}}
\renewcommand{\vec}[1]{\ensuremath{\mathbf{#1}}\xspace}
%\renewcommand{\vec}[1]{\ensuremath{\overrightarrow{#1}}}
\newcommand{\mat}[1]{\ensuremath{\mathbf{#1}}\xspace}
\newcommand{\eigenval}{\ensuremath{\lambda}\xspace}
\newcommand{\sv}{\ensuremath{\sigma}\xspace}


%%% Groups, Fields, and Rings
\newcommand{\group}{\ensuremath{\mathbb{G}}\xspace}
%\newcommand     {\G}{\group}
\newcommand{\nat}{\ensuremath{\mathbb{N}}\xspace}
\newcommand{\naturals}{\nat}
\newcommand{\N}{\nat}

\newcommand{\ints}{\ensuremath{\mathbb{Z}}\xspace}
\newcommand{\integers}{\ints}
\newcommand{\Z}{\ints}
\newcommand{\Zmod}[1]{\ensuremath{\Z_{#1}}\xspace}

\newcommand{\rat}{\ensuremath{\mathbb{Q}}\xspace}
\newcommand{\rationals}{\rat}
\newcommand{\Q}{\rat}

\newcommand{\real}{\ensuremath{\mathbb{R}}\xspace}
\newcommand{\reals}{\real}
\newcommand{\R}{\real}

\newcommand{\complex}{\ensuremath{\mathbb{C}}\xspace}
\newcommand{\C}{\complex}

\newcommand{\field}{\ensuremath{\mathbb{F}}\xspace}
\newcommand{\F}{\field}
\newcommand{\ring}{\ensuremath{\mathcal{R}}\xspace}
\newcommand{\rng}{\ring}

\newcommand{\inftypt}{\ensuremath{\mathcal{O}}\xspace}


%%% Set notation
\newcommand{\powerset}[1]{\ensuremath{\mathcal{P}(#1)}\xspace}
\newcommand{\set}[1]{\ensuremath{\left\{ #1 \right\}}\xspace}
%%% NOTE: Change if you prefer colon or vertical bar (\mid) for suchthat
%\newcommand{\suchthat}{\mathbin{:}}
\newcommand{\suchthat}{\mathbin{\mid}}
\newcommand{\given}{\suchthat}
%%% Prevents overloading strikethrough macro if the document imports soul package
\ifdefined\st\else\newcommand{\st}{\suchthat}\fi
\newcommand{\cset}[2]{\set{#1 \suchthat #2}}  % a conditional notation to define sets
\newcommand{\lset}[2]{\set{#1,\ldots,#2}} % set {from,...,to}
\renewcommand{\complement}[1]{\ensuremath{\overline{#1}}}
\newcommand{\univset}{\ensuremath{U}\xspace}

%%% Custom functions

%%% Custom operators
\newcommand{\bigcdot}{\mathbin{\bullet}}
\newcommand{\iso}{\mathbin{\cong}}
\newcommand{\grad}{\ensuremath{\nabla}}
%\newcommand{\possemidef}{\mathbin{\succeq}}
%\newcommand{\posdef}{\mathbin{\succ}}
%%% Custom equivalence Relations
\renewcommand{\approx}{\raise.17ex\hbox{\ensuremath{\scriptstyle\sim}}}
%%% Relations-checking operators
\newcommand{\iseq}{\overset{?}{=}}
\newcommand{\isequiv}{\overset{?}{\equiv}}
\newcommand{\iscong}{\overset{?}{\cong}}
\newcommand{\isiso}{\overset{?}{\iso}}
\newcommand{\isapprox}{\overset{?}{\approx}}
\newcommand{\issim}{\overset{?}{\isapprox}}
\newcommand{\issimeq}{\overset{?}{\simeq}}
\newcommand{\isdoteq}{\overset{?}{\doteq}}
\newcommand{\islt}{\overset{?}{<}}
\newcommand{\isgt}{\overset{?}{>}}
\newcommand{\isleq}{\overset{?}{\leq}}
\newcommand{\isgeq}{\overset{?}{\geq}}
\newcommand{\issubset}{\overset{?}{\subset}}
\newcommand{\issupset}{\overset{?}{\supset}}
\newcommand{\issubsetq}{\overset{?}{\subseteq}}
\newcommand{\issupsetq}{\overset{?}{\supseteq}}
\newcommand{\isprec}{\overset{?}{\prec}}
\newcommand{\ispreceq}{\overset{?}{\preceq}}
\newcommand{\issucc}{\overset{?}{\succ}}
\newcommand{\issucceq}{\overset{?}{\succeq}}
%\newcommand{\isposdef}{\overset{?}{\posdef}}
%\newcommand{\ispossemidef}{\overset{?}{\possemidef}}
\newcommand{\ispropto}{\overset{?}{\propto}}
\newcommand{\isin}{\overset{?}{\in}}


%%% Statistics
% NOTE: default for \Pr is Pr
%\DeclareMathOperator{\Prob}{P}
% NOTE: Sometimes people like to use regular E instead of Blackboard Bold
\DeclareMathOperator{\Exp}{\mathbb{E}}
%\DeclareMathOperator{\Exp}{E}
% NOTE: \Pr already exists and is normalscript math operator "Pr". I don't like that one.
\DeclareMathOperator{\Prob}{\mathbb{P}}
%\DeclareMathOperator{\Prob}{P}
\DeclareMathOperator{\Var}{Var}
\DeclareMathOperator{\Covar}{Cov}
\DeclareMathOperator*{\Max}{\max}
\DeclareMathOperator*{\Min}{\min}
% NOTE: This is the same thing as the Universal Set
\newcommand{\samplespace}{\ensuremath{\Omega}\xspace}
\newcommand{\eventspace}{\ensuremath{\mathcal{F}}\xspace}
\newcommand{\probspace}{\ensuremath{(\samplespace, \eventspace, \Prob)}\xspace}
\newcommand{\sample}{\ensuremath{\omega}\xspace}
\newcommand{\event}{\ensuremath{E}\xspace}
\newcommand{\cdf}{\ensuremath{F}\xspace}
% NOTE: Technically this is pdf in continuous land but meh
\newcommand{\pmf}{\ensuremath{p}\xspace}
% NOTE: This is likely to conflict (both in macro and notation) with other things, e.g. digital signatures or state.
\newcommand{\sd}{\ensuremath{\sigma}\xspace}
%\newcommand{\sd}{\ensuremath{\var{sd}}\xspace}
%\newcommand{\corr}{\ensuremath{\rho}\xspace}
\newcommand{\corr}{\ensuremath{r}\xspace}
\newcommand{\covarmat}{\ensuremath{\Sigma}\xspace}
\newcommand{\likelihood}{\ensuremath{\mathcal{L}}\xspace}
%%% Calculus
\newcommand{\approaches}{\mathbin{\rightarrow}}

%%%%%%%%%%%%%%%%%%%
% CS Macros       %
%%%%%%%%%%%%%%%%%%%
%%% General
\newcommand{\alg}[1]{\ensuremath{\mathsf{#1}}\xspace}
\newcommand{\cmd}[1]{\texttt{#1}\xspace}
\newcommand{\var}[1]{\ensuremath{\mathsf{#1}}\xspace}
%\newcommand{\true}{\var{T}}
%\newcommand{\false}{\var{F}}
%\newcommand{\boolset}{\set{\true,\false}}
\newcommand{\true}{1}
\newcommand{\false}{0}
\newcommand{\boolset}{\bitset}
\renewcommand{\implies}{\mathbin{\longrightarrow}}
\newcommand{\bitset}{\set{0,1}}
\newcommand{\bitsset}[1][*]{\ensuremath{\bitset^{#1}}\xspace}
\newcommand{\addeq}{\ensuremath{+=}\xspace}
\newcommand{\subeq}{\ensuremath{-=}\xspace}
\newcommand{\muleq}{\ensuremath{*=}\xspace}
\newcommand{\diveq}{\ensuremath{/=}\xspace}
\renewcommand{\gets}{\coloneqq}
\newcommand{\assigneq}{\gets}
\newcommand{\nil}{\ensuremath{\var{nil}}\xspace}
\newcommand{\error}{\ensuremath{\bot}\xspace}
\newcommand{\ignored}{{\ensuremath{\,\cdot\,}}}
\newcommand{\codecomment}[1]{{\footnotesize // #1}}
\newcommand{\varlist}[1]{\ensuremath{\overline{#1}}\xspace}

%%% Complexity Theory
\newcommand{\mapreducesto}{\leq_m}
\newcommand{\mapreducesfrom}{\geq_m}
\newcommand{\mapequiv}{\equiv_m}
\newcommand{\polyreducesto}{\leq_P}
\newcommand{\polyreducesfrom}{\geq_P}
\newcommand{\polyequiv}{\equiv_P}
\newcommand{\logspacereducesto}{\leq_L}
\newcommand{\logspacereducesfrom}{\geq_L}
\newcommand{\logspaceequiv}{\equiv_L}
%%% Computation theory
\newcommand{\yields}{\rightarrow_M}
\newcommand{\yieldsto}[1][*]{\overset{#1}{\rightarrow}_M}
\newcommand{\blanksymbol}{\ensuremath{%
  \ooalign{%
    \relax\cr%
    \noalign{\vskip.2ex}%
    \textvisiblespace\cr%
    \noalign{\vskip-.2ex}%
    \hphantom{~}\cr%   Width of a space character
    \vphantom{Xg}\cr}%  Height of X, depth of g
  }\xspace%
}
\newcommand{\desc}[1]{\ensuremath{\left\langle #1 \right\rangle}\xspace}
%%% Formal Syntax + Grammars + Languages
\newcommand{\derives}{\mathbin{\rightarrow}}
\newcommand{\reduces}{\derives}
\newcommand{\derivesto}[1][*]{\mathbin{\overset{#1}{\rightarrow}}}
\newcommand{\reducesto}{\derivesto}
\newcommand{\partreducesto}{\mathbin{\rightharpoonup}}
\newcommand{\evaluatesto}{\mathbin{\Downarrow}}
\newcommand{\subtypeof}{\mathbin{<:}}
\newcommand{\assumedby}{\mathbin{\vdash}}
\newcommand{\syntaxeq}{\mathbin{::=}}
\newcommand{\state}{\ensuremath{\sigma}\xspace}
\newcommand{\heap}{\ensuremath{\heta}\xspace}
\newcommand{\locmap}{\ensuremath{\zeta}\xspace}
%%% From: https://tex.stackexchange.com/questions/41357/replace-cmtt10-char32-visible-cup-space-with-something-more-gentle/41445#41445
\newcommand{\emptystring}{\ensuremath{\epsilon}\xspace}
\newcommand{\emptysymbol}{\emptystring}
\newcommand{\langconcat}{\mathbin{\circ}}
\newcommand{\any}[1]{{#1}^*}
\newcommand{\kleenestar}[1]{\any{#1}}
\newcommand{\strconcat}{\mathbin{\Vert}}
\newcommand{\concat}{\strconcat}
%%% Graphs
\newcommand{\pathto}{\mathbin{\rightsquigarrow}}
\newcommand{\dirpath}{\pathto}
\newcommand{\dirpathto}{\pathto}
\newcommand{\undirpath}{\mathbin{\leftrightsquigarrow}}
\newcommand{\undirpathto}{\undirpath}
\newcommand{\undirpathfrom}{\undirpath}
\newcommand{\walkto}{\mathbin{\looparrowright}}
\newcommand{\walkfrom}{\mathbin{\looparrowleft}}
\newcommand{\indeg}{\deg_{-}}
\newcommand{\outdeg}{\deg_{+}}
%%% Cryptography
\newcommand{\secparam}{\ensuremath{\lambda}\xspace}
\newcommand{\crs}{\var{crs}}
\newcommand{\rel}[1]{\ensuremath{R_{\mathsf{#1}}}\xspace}
% TODO: Think of something better
\newcommand{\com}{\ensuremath{\var{com}}\xspace}
\newcommand{\sign}{\ensuremath{\sigma}\xspace}
\newcommand{\hashfn}{\ensuremath{\mathsf{H}}\xspace}
\newcommand{\pf}{\ensuremath{\pi}\xspace}
\newcommand{\user}{\ensuremath{U}\xspace}
\newcommand{\verifier}{\ensuremath{V}\xspace}
\newcommand{\simulator}{\ensuremath{\mathcal{S}}\xspace}
\newcommand{\adversary}{\ensuremath{\mathcal{A}}\xspace}
\newcommand{\aux}{\ensuremath{\var{aux}}\xspace}
\newcommand{\context}{\ensuremath{\var{ctx}}\xspace}
\newcommand{\pairingparams}{\ensuremath{\var{pp}}\xspace}


%%%%%%%%%%%%%%%%%%%
% Theorems etc.   %
%%%%%%%%%%%%%%%%%%%
\newcounter{problem}
\renewcommand{\theproblem}{\arabic{problem}}    
\newenvironment{problem}[1][] { % Optionally include descriptor and make following text italics by default
    \stepcounter{problem} \noindent \underline{\textbf{Problem \theproblem}}:
    \ifx\newenvironment#1\newenvironment\else\space(#1)\fi \space\itshape 
} {\vspace{1em}}

\newenvironment{solution} {\par} {\newpage}
%\renewenvironment{proof} {\vspace{0.5em} \noindent \underline{\textbf{Pf}}:} {\hfill \qedsymbol \vspace{0.5em}}      % Default proof, but with _*Pf:*_ instead
\newenvironment{theoremproof} {\noindent \textit{Proof \thetheorem.}} {\hfill \qedsymbol \vspace{0.5em}}      % Default proof, numbered by theorem
\newenvironment{corollaryproof} {\noindent \textit{Proof \thecorollary.}} {\hfill \qedsymbol \vspace{0.5em}}      % Default proof, numbered by corollary
\newenvironment{lemmaproof} {\noindent \textit{Proof \thelemma.}} {\hfill \qedsymbol \vspace{0.5em}}      % Default proof, numbered by lemma
\newenvironment{proofsketch} {\noindent \textit{Proof (Sketch).}} {\vspace{0.5em}}      

%%% Default proofs, but as an informal description
\ifdefined\definition\else\newtheorem{definition}{Definition}\fi
\newtheorem*{definition*}{Definition}
%%% NOTE: use "problem" if HW, "theorem" if generic, or "section" if paper
\ifdefined\theorem\else\newtheorem{theorem}{Theorem}[problem]\fi
\newtheorem*{theorem*}{Theorem}
%%% NOTE: use "problem" if HW, "theorem" if generic, or "section" if paper
\ifdefined\assumption\else\newtheorem{assumption}{Assumption}[problem]\fi
\newtheorem*{assumption*}{Assumption}
\ifdefined\corollary\else\newtheorem{corollary}{Corollary}[theorem]\fi
\newtheorem*{corollary*}{Corollary}
\ifdefined\lemma\else\newtheorem{lemma}{Lemma}[theorem]\fi
\newtheorem*{lemma*}{Lemma}
%%% NOTE: use "problem" if HW, "theorem" if generic, or "section" if paper
\ifdefined\claim\else\newtheorem{claim}{Claim}[problem]\fi
\newtheorem*{claim*}{Claim}
\ifdefined\observation\else\newtheorem*{observation}{Observation}\fi
\ifdefined\proposition\else\newtheorem*{proposition}{Proposition}\fi
\ifdefined\fact\else\newtheorem*{fact}{Fact}\fi
\ifdefined\remark\else\newtheorem*{remark}{Remark}\fi

% Modifies the enumerate* to use 1) 2) 3) by default
\newlist{enumerate*}{enumerate*}{1}
\setlist[enumerate*]{label=\arabic*)}

%%%%%%%%%%%%%%%%%%%
% Text shortcuts  %
%%%%%%%%%%%%%%%%%%%
\renewcommand{\th}{\textsuperscript{th}\xspace}
\newcommand{\wrt}{\xspace{with respect to}\xspace}
\newcommand{\Wrt}{\xspace{With respect to}\xspace}
%%% NOTE: \wlog is a logging command and can't be redefined
\newcommand{\WLOG}{\xspace{without loss of generality}\xspace}
\newcommand{\Wlog}{\xspace{Without loss of generality}\xspace}
\newcommand{\str}[1]{\ifmmode{\mathstrut\text{#1}}\else\mbox{\strut#1}\fi}
\newcommand{\TODO}[1]{\par \noindent \textbf{\underline{TODO:}}\space#1}
\newcommand{\NOTE}[1]{\par \noindent \textbf{NOTE:}\space#1}